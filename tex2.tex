\documentclass[a4paper,12pt]{jarticle}

%%%
%  パッケージの宣言
%%%
\usepackage[dvipdfmx]{graphicx}      % パッケージ graphicx の利用宣言
                                     % dvipdfmx は pdfファイル挿入のためのオプション
\usepackage{amsmath}                 % パッケージ amsmath の利用宣言
\usepackage{amssymb}                 % パッケージ amssymb の利用宣言
\usepackage{amsthm}                  % パッケージ amsthm の利用宣言
\usepackage{bm}                      % パッケージ bm の利用宣言

%%%
%  ページレイアウトの設定
%%%
\setlength{\textwidth}{15cm}         % テキストの幅を15cmに設定
\setlength{\textheight}{24cm}        % テキストの高さを24cmに設定
\setlength{\oddsidemargin}{0.5cm}    % 左余白を1インチ(2.54cm)+\oddsidemarginと設定
\setlength{\topmargin}{-1.0cm}       % 上余白を1インチ(2.54cm)+\topmarginと設定
\renewcommand{\baselinestretch}{1.2} % すべての文字サイズの行送りを1.2倍と設定
%\pagestyle{bothstyle}                % ヘッダーに見出し,フッターにページ番号を出力

%%%
%  個人用マクロの設定
%%%
%\theoremstyle{definition}            % 定理環境内の英数字がイタリックでなくなる
\newtheorem{theorem}{定理}[section]
\newtheorem{proposition}[theorem]{命題}
\newtheorem{remark}{註}[section]
%
\newcommand{\combi}[2]{{}_{#1}C_{#2}}
\DeclareMathOperator*{\argmax}{\mathrm{arg\,max}}

\begin{document}

\title{LaTeX レポート課題 第2回}
\author{横井 暉 \thanks{\texttt{pandorabox0720@keio.jp}}\\
(61920820)}
\date{2020年12月18日}
%
\maketitle
%\thispagestyle{empty}                 % タイトルページにページ番号を付けないためのコマンド

\begin{abstract}

この演習では, 複雑な数式の書き方と図・表の書き方を学ぶ.

\end{abstract}

\section{数式の書き方(続き)}

\begin{enumerate}

\item 二次方程式$ax^2 + bx + c = 0$ $(a \neq 0)$ の解は,
\[ x = \frac{ -b \pm \sqrt{b^2 - 4ac}}{2a} \]
である.

\item \textbf{(Taylorの定理)}
$f (x)$が$a \leq x \leq b$で$n$回微分可能であるとき,\\
\begin{align*}
f(b) &=  f(a) + \frac{f'(a)}{1!}(b - a) + \ldots + \frac{f^{(n-1)}(a)}{(n-1)!} + \frac{f^{(n)}(c)}{n!}(b-a)^n\\
     &= \sum_{k = 0}^{n - 1} f^{(k)}(a) \frac{(b-a)^k}{k!} + \frac{f^{(n)}(c)}{n!} (b-a)^n
\end{align*} 
を満たす点$c$が$a$と$b$の間に存在する.
\item \textbf{(ガウス積分)}
ガウス関数$y = e^{-x^2}$のグラフを書くと図$1$のようになる. また, ガウス関数について,
\[ \int_{-\infty}^{\infty} e^{-x^2} dx = \sqrt{\pi\\} \]
が成り立つ.
\begin{figure}[h] \centering
\includegraphics[width=50mm, clip]{Gauss.pdf}
\caption{ガウス関数}\label{figure1}
\end{figure}
\end{enumerate}

\section{フィボナッチ数列}
フィボナッチ数列$\{f_k \}$は以下の$3$項間漸化式で定義される.
\begin{align*} f_0 &= f_1 = 1,\\
f_i &= f_{i-1} +f_{i-2} \,\,\text{($i \geq 2$のとき).}
\end{align*}
小さな$k$でのフィボナッチ数を計算してみよう. $f_0,f_1,\ldots,f_9$を(手や電卓で)計算すると, 表1が得られる.
\begin{table}[h] \centering
\caption{フィボナッチ数の例} 
\vspace{0.5cm}
\begin{tabular}{c||c|c|c|c|c}
$k$ & $0$ & $1$ & $2$ & $3$ & $4$ \\ \hline
$f_k$ & $1$ & $1$ & $2$ & $3$ & $5$   \\ \hline \hline
$k$ & $5$ & $6$ & $7$ & $8$ & $9$ \\ \hline
$f_k$ & $8$ & $13$ & $21$ & $34$ & $55$ 
\end{tabular}
\end{table}

次に$f_k$の一般項を計算してみよう. $k \geq 1$に対して, $f_{k+1},f_{k}$を並べて書くと
\begin{equation} 
\begin{pmatrix} 
f_{k+1} \\ f_k 
\end{pmatrix} 
= \begin{pmatrix} 1 & 1 \\ 1 & 0 \end{pmatrix}
\begin{pmatrix} f_k \\ f_{k-1} \end{pmatrix}
\end{equation}
という関係がある. 行列$\left( \begin{smallmatrix} 1 & 1 \\ 1 & 0 \end{smallmatrix} \right)$を$A$とおき, ($1$)を繰り返し適用すると,
\begin{equation} 
\begin{pmatrix} f_{k+1} \\ f_k \end{pmatrix} = A \begin{pmatrix} f_k \\ f_{k-1} \end{pmatrix} = \cdots 
= A^k \begin{pmatrix} f_1 \\ f_0 \end{pmatrix} = A^k \begin{pmatrix} 1 \\ 1 \end{pmatrix} 
\end{equation}
が成り立つ. したがって$f_k$は$A^k \left( \begin{smallmatrix} 1 \\ 1 \end{smallmatrix} \right) $の第2成分と等しい.

行列$A$のべき乗の計算には以下の定理2.1を利用する (\cite{k}参照).

\begin{theorem} \label{thm2.1}
$n$次正方行列$M$の固有値$\lambda_1$,$\lambda_2$, \ldots ,$\lambda_n$ が全て異なると仮定する. $\lambda_i$に関する固有ベクトルを
$\bm{x_i}$とおき,  $\bm{x_1}$,$\bm{x_2}$, \ldots ,$\bm{x_n}$ を並べた行列を$U$とする. つまり,
\[U = \Big[ \,\ \bm{x_1} \,\ \bm{x_2} \,\ \cdots \,\ \bm{x_n} \,\ \Big] \]
である. このとき, $U$は正則であり,
\[ M = U \left( \begin{array} {ccccc} 
\lambda_1  & 0 & 0 & \cdots & 0 \\
0 & \lambda_2 & 0 & \ddots & 0 \\
0 & 0 & \ddots & \ddots & \vdots \\
\vdots & \ddots & \ddots & \lambda_{n-1} & 0 \\
0 & 0 & \cdots & 0 & \lambda_{n}  \end{array}\right) U^{-1} \]
が成り立つ.
\end{theorem}

まず行列$A$の固有値を計算しよう. 固有多項式は,
\[ \det (A - \lambda I ) = \lambda^2 - \lambda - 1 = 0 \]
となる. したがって, 固有値$\lambda_1$,$\lambda_2$ ($\lambda_1 \geq \lambda_2$)は,
\[\lambda_1 = \frac{1 + \sqrt{5}} {2}, \,\ \lambda_2 = \frac{1 - \sqrt{5}} {2} \]
であり, 対応する固有ベクトルは
\[ \bm{x_1} = \begin{pmatrix} 1 \\ -\lambda_2 \end{pmatrix}, \,\ \bm{x_2} = \begin{pmatrix} 1 \\ -\lambda_1 \end{pmatrix} \]
となる. ここで,
\[ U = \Big[ \,\ \bm{x_1} \,\ \bm{x_2} \,\ \Big] 
= \left( \begin{array} {cc} 1 & 1 \\ -\lambda_2 & -\lambda_1 \end{array} \right) \]
とおくと,
\[ U^{-1} = \frac{1}{\lambda_1 - \lambda_2} \left( \begin{array} {cc}
\lambda_1 & 1 \\ -\lambda_2 & -1 \end{array} \right)  \]
であり, 定理2.1より,
\[ A = U \begin{pmatrix} \lambda_1 & 0 \\ 0 & \lambda_2 \end{pmatrix} U^{-1} \]
と書ける. $\Lambda = \begin{pmatrix} \lambda_1 & 0 \\ 0 & \lambda_2 \end{pmatrix} $ とおくと,
\begin{align*} A^k &= (U \Lambda U^{-1}) (U \Lambda U^{-1}) \cdots (U \Lambda U^{-1}) \\
& = U \Lambda^k U^{-1} \end{align*}
が成り立つ. (3)を(2)に代入して,
\begin{align*} \begin{pmatrix} f_{k+1} \\ f_k \end{pmatrix}
&= U \begin{pmatrix} \lambda_1^k & 0 \\ 0 & \lambda_2^k \end{pmatrix} U^{-1}  \begin{pmatrix} 1 \\ 1 \end{pmatrix} \\
&= \left( \begin{array} {cc} 1 & 1 \\ -\lambda_2 & -\lambda_1 \end{array} \right)
\begin{pmatrix} \lambda_1^k & 0 \\ 0 & \lambda_2^k \end{pmatrix}
\frac{1}{\lambda_1 - \lambda_2} \left( \begin{array} {cc} \lambda_1 & 1 \\ -\lambda_2 & -1 \end{array} \right)
\begin{pmatrix} 1 \\ 1 \end{pmatrix} \\
&= \frac{1}{\lambda_1 - \lambda_2} \begin{pmatrix} \lambda_1^k & \lambda_2^k \\ -\lambda_2 \lambda_1^k & -\lambda_1 \lambda_2^k \end{pmatrix}
\begin{pmatrix} \lambda_1 + 1 \\ -\lambda_2 - 1 \end{pmatrix} \\
&=  \frac{1}{\lambda_1 - \lambda_2} 
\begin{pmatrix} \lambda_1^k (\lambda_1 + 1) + \lambda_2^k ( -\lambda_2 - 1) \\ 
-\lambda_2\lambda_1^k (\lambda_1 + 1) - \lambda_1\lambda_2^k ( -\lambda_2 - 1) \end{pmatrix} \\
&=  \frac{1}{\lambda_1 - \lambda_2} \begin{pmatrix} \lambda_1^k ( \lambda_1 + 1) - \lambda_2^k ( \lambda_2 + 1) \\ 
-\lambda_2 \lambda_1^k (\lambda_1 + 1) + \lambda_1 \lambda_2^k (\lambda_2 + 1) \end{pmatrix} 
\end{align*}
と計算できる. $\lambda_1$と$\lambda_2$が満たす関係

\[ \lambda_i^2 = \lambda_i + 1\,\ (i = 1,2), \,\ \lambda_1\lambda_2 = -1, \,\ \ \lambda_1 + \lambda_2 = 1, 
\,\ \lambda_1 - \lambda_2 = \sqrt{5} \]
などを用いて整理すると,最終的に
\[ f_k = \frac{1}{\sqrt{5}} ( \lambda_1^{k+1} - \lambda_2^{k+1} ) \]
を得る.


%%%
%  参考文献の登録
%%%
%\begin{thebibliography}
%\bibitem
%\end{thebibliography}


\begin{thebibliography}{9}
\bibitem{k} 慶應義塾大学理工学部数理科学科編, 数学2・4, 2017.
\end{thebibliography}

\end{document}

